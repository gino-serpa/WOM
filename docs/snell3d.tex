\documentclass[10pt,a4paper]{article}
\usepackage[utf8]{inputenc}
\usepackage{amsmath}
\usepackage{amsfonts}
\usepackage{amssymb}
\begin{document}
Since the refracted ray should be in the same plane as the incident ray and the normal we should be able to express it as:

\[
r_r = \alpha  n + \beta r_i
\]

$\alpha$ and $\beta$ can be obtained from the noting that $r$ is a unit vector 
and Snell's law:

\begin{align*}
r_i \cdot r_i                                         & = 1 \\
(\alpha  n + \beta r_i) \cdot (\alpha  n + \beta r_i) & = 1 \\
\alpha^2 + \beta^2 + 2 \alpha \beta (n \cdot r_i)     & = 1 
\end{align*}


and from Snell

\[
n_i \sin{\theta_i} = n_r \sin{\theta_r}
\Rightarrow
\theta_r = \arcsin{ \left( \dfrac{n_i}{n_r}\sin{\theta_i} \right) }
\]

where

\[
\theta_i = \arccos{\left( -n \cdot r_i \right)}
\]

And the expression ofr the angle on the refracted side:

\begin{align*}
-n \cdot r_r                     & = \cos\theta_r  \\
(-n)\cdot (\alpha n +\beta r_i)  & = \cos\theta_r  \\
-\alpha-\beta (n\cdot r_i)       & = \cos\theta_r  \\
\alpha                           & = -\cos\theta_r -\beta(n \cdot r_i) 
\end{align*}


replacing this in the first equation

\[
(- \cos\theta_r - \beta (n \cdot r_i))^2+ \beta^2 +2(-\cos\theta_r-\beta(n\cdot r_i))\beta(n\cdot r_i)=1
\]

\[
(1-(n\cdot r_i)^2)\beta^2 = 1-\cos^2\theta_r
\]

\[
\beta = + \sqrt{\dfrac{1-\cos^2\theta_r}{1-(n \cdot r_i)^2}}
\]




\end{document}